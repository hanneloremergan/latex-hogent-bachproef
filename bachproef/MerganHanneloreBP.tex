%===============================================================================
% LaTeX sjabloon voor de bachelorproef toegepaste informatica aan HOGENT
% Meer info op https://github.com/HoGentTIN/latex-hogent-report
%===============================================================================

\documentclass[dutch,dit,thesis]{hogentreport}

% TODO:
% - If necessary, replace the option `dit`' with your own department!
%   Valid entries are dbo, dbt, dgz, dit, dlo, dog, dsa, soa
% - If you write your thesis in English (remark: only possible after getting
%   explicit approval!), remove the option "dutch," or replace with "english".

\usepackage{lipsum} % For blind text, can be removed after adding actual content

%% Pictures to include in the text can be put in the graphics/ folder
\graphicspath{{../graphics/}}

%% For source code highlighting, requires pygments to be installed
%% Compile with the -shell-escape flag!
%% \usepackage[chapter]{minted}
%% If you compile with the make_thesis.{bat,sh} script, use the following
%% import instead:
\usepackage[chapter,outputdir=../output]{minted}
\usemintedstyle{solarized-light}

%% Formatting for minted environments.
\setminted{%
    autogobble,
    frame=lines,
    breaklines,
    linenos,
    tabsize=4
}

%% Ensure the list of listings is in the table of contents
\renewcommand\listoflistingscaption{%
    \IfLanguageName{dutch}{Lijst van codefragmenten}{List of listings}
}
\renewcommand\listingscaption{%
    \IfLanguageName{dutch}{Codefragment}{Listing}
}
\renewcommand*\listoflistings{%
    \cleardoublepage\phantomsection\addcontentsline{toc}{chapter}{\listoflistingscaption}%
    \listof{listing}{\listoflistingscaption}%
}

% Other packages not already included can be imported here

%%---------- Document metadata -------------------------------------------------
% TODO: Replace this with your own information
\author{Hannelore Mergan}
\supervisor{Mevr. L. Van Steenberghe}
\cosupervisor{Dhr. A. Willekens}
\title[Een benadering met gesynthetiseerde en reële data]%
    {Semantisch segmentatiemodel voor gewasdetectie in wijngaarden}
\academicyear{\advance\year by -1 \the\year--\advance\year by 1 \the\year}
\examperiod{1}
\degreesought{\IfLanguageName{dutch}{Professionele bachelor in de toegepaste informatica}{Bachelor of applied computer science}}
\partialthesis{false} %% To display 'in partial fulfilment'
%\institution{Internshipcompany BVBA.}

%% Add global exceptions to the hyphenation here
\hyphenation{back-slash}

%% The bibliography (style and settings are  found in hogentthesis.cls)
\addbibresource{bachproef.bib}            %% Bibliography file
\addbibresource{../voorstel/voorstel.bib} %% Bibliography research proposal
\defbibheading{bibempty}{}

%% Prevent empty pages for right-handed chapter starts in twoside mode
\renewcommand{\cleardoublepage}{\clearpage}

\renewcommand{\arraystretch}{1.2}

%% Content starts here.
\begin{document}

%---------- Front matter -------------------------------------------------------

\frontmatter

\hypersetup{pageanchor=false} %% Disable page numbering references
%% Render a Dutch outer title page if the main language is English
\IfLanguageName{english}{%
    %% If necessary, information can be changed here
    \degreesought{Professionele Bachelor toegepaste informatica}%
    \begin{otherlanguage}{dutch}%
       \maketitle%
    \end{otherlanguage}%
}{}

%% Generates title page content
\maketitle
\hypersetup{pageanchor=true}

\input{voorwoord}
\input{samenvatting}

%---------- Inhoud, lijst figuren, ... -----------------------------------------

\tableofcontents

% In a list of figures, the complete caption will be included. To prevent this,
% ALWAYS add a short description in the caption!
%
%  \caption[short description]{elaborate description}
%
% If you do, only the short description will be used in the list of figures

\listoffigures

% If you included tables and/or source code listings, uncomment the appropriate
% lines.
\listoftables

\listoflistings

% Als je een lijst van afkortingen of termen wil toevoegen, dan hoort die
% hier thuis. Gebruik bijvoorbeeld de ``glossaries'' package.
% https://www.overleaf.com/learn/latex/Glossaries

%---------- Kern ---------------------------------------------------------------

\mainmatter{}

% De eerste hoofdstukken van een bachelorproef zijn meestal een inleiding op
% het onderwerp, literatuurstudie en verantwoording methodologie.
% Aarzel niet om een meer beschrijvende titel aan deze hoofdstukken te geven of
% om bijvoorbeeld de inleiding en/of stand van zaken over meerdere hoofdstukken
% te verspreiden!

%%=============================================================================
%% Inleiding
%%=============================================================================

\chapter{\IfLanguageName{dutch}{Inleiding}{Introduction}}%
\label{ch:inleiding}

De inleiding moet de lezer net genoeg informatie verschaffen om het onderwerp te begrijpen en in te zien waarom de onderzoeksvraag de moeite waard is om te onderzoeken. In de inleiding ga je literatuurverwijzingen beperken, zodat de tekst vlot leesbaar blijft. Je kan de inleiding verder onderverdelen in secties als dit de tekst verduidelijkt. Zaken die aan bod kunnen komen in de inleiding~\autocite{Pollefliet2011}:

\begin{itemize}
  \item context, achtergrond
  \item afbakenen van het onderwerp
  \item verantwoording van het onderwerp, methodologie
  \item probleemstelling
  \item onderzoeksdoelstelling
  \item onderzoeksvraag
  \item \ldots
\end{itemize}

Twintig jaar geleden was er nauwelijks sprake van Belgische wijnbouw. De laatste jaren is deze sector echter uitgegroeid tot een productieve pijler binnen onze landbouw. Volgens \textcite{FODEconomie2024} werd er in 2023 meer dan 3,4 miljoen liter wijn geproduceerd, en die groei zet zich verder voort. 

(FAIR, ILVO introduceren / kader van stageplaats)

\section{\IfLanguageName{dutch}{Probleemstelling}{Problem Statement}}%
\label{sec:probleemstelling}

%Uit je probleemstelling moet duidelijk zijn dat je onderzoek een meerwaarde heeft voor een concrete doelgroep. De doelgroep moet goed gedefinieerd en afgelijnd zijn. Doelgroepen als ``bedrijven,'' ``KMO's'', systeembeheerders, enz.~zijn nog te vaag. Als je een lijstje kan maken van de personen/organisaties die een meerwaarde zullen vinden in deze bachelorproef (dit is eigenlijk je steekproefkader), dan is dat een indicatie dat de doelgroep goed gedefinieerd is. Dit kan een enkel bedrijf zijn of zelfs één persoon (je co-promotor/opdrachtgever).

De autonome landbouwrobots bij ILVO moeten hun omgeving visueel kunnen waarnemen om zelfstandig tussen de wijnranken te navigeren. Daarvoor is computervisie nodig, zodat ze gewassen en obstakels kunnen herkennen en hierop reageren.

Het semantische segmentatiemodel moet in realtime de druivenranken en -trossen kunnen segmenteren. Het implementeren van dergelijke modellen op edge devices, zoals landbouwrobots, vormt een uitdaging vanwege beperkte rekenkracht en geheugen. Daarnaast presteren deze modellen vaak minder goed in een landbouwcontext, omdat hun architectuur niet is afgestemd op de complexiteit en variabiliteit van natuurlijke omgevingen. Het grootste struikelblok blijft echter het beperkte aantal beschikbare (gelabelde) landbouwdatasets, aangezien het annoteren van vegetatiebeelden een zeer tijdrovend en arbeidsintensief proces is. Dit bemoeilijkt de training en generalisatie van het model aanzienlijk.

\section{\IfLanguageName{dutch}{Onderzoeksvraag}{Research question}}%
\label{sec:onderzoeksvraag}

Wees zo concreet mogelijk bij het formuleren van je onderzoeksvraag. Een onderzoeksvraag is trouwens iets waar nog niemand op dit moment een antwoord heeft (voor zover je kan nagaan). Het opzoeken van bestaande informatie (bv. ``welke tools bestaan er voor deze toepassing?'') is dus geen onderzoeksvraag. Je kan de onderzoeksvraag verder specifiëren in deelvragen. Bv.~als je onderzoek gaat over performantiemetingen, dan 

\section{\IfLanguageName{dutch}{Onderzoeksdoelstelling}{Research objective}}%
\label{sec:onderzoeksdoelstelling}

Wat is het beoogde resultaat van je bachelorproef? Wat zijn de criteria voor succes? Beschrijf die zo concreet mogelijk. Gaat het bv.\ om een proof-of-concept, een prototype, een verslag met aanbevelingen, een vergelijkende studie, enz.

\section{\IfLanguageName{dutch}{Opzet van deze bachelorproef}{Structure of this bachelor thesis}}%
\label{sec:opzet-bachelorproef}

% Het is gebruikelijk aan het einde van de inleiding een overzicht te
% geven van de opbouw van de rest van de tekst. Deze sectie bevat al een aanzet
% die je kan aanvullen/aanpassen in functie van je eigen tekst.

De rest van deze bachelorproef is als volgt opgebouwd:

In Hoofdstuk~\ref{ch:stand-van-zaken} wordt een overzicht gegeven van de stand van zaken binnen het onderzoeksdomein, op basis van een literatuurstudie.

In Hoofdstuk~\ref{ch:methodologie} wordt de methodologie toegelicht en worden de gebruikte onderzoekstechnieken besproken om een antwoord te kunnen formuleren op de onderzoeksvragen.

% TODO: Vul hier aan voor je eigen hoofstukken, één of twee zinnen per hoofdstuk

In Hoofdstuk~\ref{ch:conclusie}, tenslotte, wordt de conclusie gegeven en een antwoord geformuleerd op de onderzoeksvragen. Daarbij wordt ook een aanzet gegeven voor toekomstig onderzoek binnen dit domein.
\chapter{\IfLanguageName{dutch}{Stand van zaken}{State of the art}}%
\label{ch:stand-van-zaken}

% 10-12 pagina's

% "De student levert eigen bijdrage, heeft informatie verzameld van bestaande kennis en hier relevante inzichten voor de OV uit geëxtraheerd die de student expliciet linkt aan de OV en deelvragen van probleem- en oplossingsdomein. + De student gaat in de literatuurstudie in op zowel probleemdomein als oplossingsdomein."

% ZIE BRONNEN EN TEKST VOORSTEL + zie eigen notities !!!

% Tip: Begin elk hoofdstuk met een paragraaf inleiding die beschrijft hoe
% dit hoofdstuk past binnen het geheel van de bachelorproef. Geef in het
% bijzonder aan wat de link is met het vorige en volgende hoofdstuk.
% Pas na deze inleidende paragraaf komt de eerste sectiehoofding.

\subsection*{Slimme landbouw}
Slimme landbouw is een vakgebied in sterke ontwikkeling. Volgens een analyse van \textcite{Karunathilake2023} transformeren technologische innovaties het traditionele landbouwlandschap ingrijpend. De digitalisering van de agricultuur opent nieuwe mogelijkheden en kansen, en deze evolutie versnelt met toepassingen en benaderingen die voortdurend verfijnd worden. Het beheer van gewassen, grondstoffen en processen krijgt een moderne omkadering, met als doel de steeds groeiende vraag naar voedsel bij te houden.

De voordelen zijn aanzienlijk: door een hogere efficiëntiegraad van grondstoffen en arbeid kunnen landbouwbedrijven hun ecologische voetafdruk verkleinen en doeltreffender opereren. De efficiëntie wordt voornamelijk bereikt door het verbeteren van twee kernaspecten. Deze zijn het verzamelen van data en het omzetten van deze inzichten in acties op het veld.

Via geavanceerde sensoren en intelligente monitoringtools krijgen landbouwers realtime data, waaronder informatie over de gewasgezondheid en -bescherming. Technologieën zoals Internet of Things, robotica en computervisie automatiseren arbeidsintensieve processen. Hierdoor worden oogstopbrengst en productiviteit geoptimaliseerd.

Deze bachelorproef situeert zich binnen deze context, met een bijzondere focus op computervisie.

\section{Computervisie in precisielandbouw}

% Inleiding schrijven

\subsection{Wat is precisielandbouw}
Deze bachelorproef kan binnen het grotere geheel van precisielandbouw worden geplaatst. \textcite{Cisternas2020} definiëren dit als volgt: 

\begin{tcolorbox}[colback=gray!5, colframe=white, sharp corners, boxrule=0pt, width=\linewidth]
    \textit{“Precisielandbouw is een managementstrategie die gebruik maakt van IT om nuttige gegevens uit verschillende bronnen te verzamelen ter ondersteuning van beslissingen met betrekking tot de productie van gewassen. Het bestaat uit het herkennen, lokaliseren, kwantificeren en registreren van de ruimtelijke en temporele variabiliteit van elke landbouweenheid.”} 
\end{tcolorbox}

De auteurs benadrukken dat computervisie een veelbelovende technologie is binnen dit domein. Door geavanceerde beeldverwerking en machinelearning biedt deze technologie nieuwe mogelijkheden voor landbouwtoepassingen. Beeldsensoren en algoritmen kunnen bijvoorbeeld helpen bij het detecteren van ziektes, het analyseren van gewasgroei en het automatisch identificeren van onkruid. Dit draagt onder andere bij aan een verminderd gebruik van pesticiden.

\subsection{Wat is computervisie}
\textcite{Radojcic2023} beschrijven computervisie als een moderne technologie met de capaciteit om objecten te volgen, lokaliseren en identificeren. Het stelt machines in staat visuele informatie te ontvangen en verwerken, vergelijkbaar met mensen. Camera’s en sensoren geven robots de mogelijkheid hun omgeving waar te nemen in de vorm van pixels. Hiervoor worden algoritmen voortdurend verbeterd om de beelden zo nauwkeurig mogelijk te interpreteren. 

Dit vakgebied wordt breed onderzocht en ingezet bij verschillende sectoren. Het vindt toepassingen in de industrie, medische wereld en precisielandbouw. Voorbeelden zijn zelfrijdende auto’s, defectdetectie op lopende banden en röntgenbeeldverwerking. Binnen de landbouw wordt computervisie ingezet voor taken zoals gewasmonitoring en geautomatiseerd oogsten.

\subsection{Applicaties van computervisie}
Verder stellen \textcite{Radojcic2023} dat computervisie het proces van beoordeling, sortering en kwaliteitscontrole van fruit en groenten automatiseert. Met behulp van computervisiemodellen kunnen gewassen geclassificeerd worden op basis van grootte, vorm, kleur, textuur en andere kenmerken. De hoogwaardige gewassen worden sneller verwerkt en behouden een hogere versheid tegen de tijd dat ze in de handel komen.

Daarnaast detecteren camera’s met computervisie vroegtijdig signalen van plantenziektes en insectenplagen. Hierdoor kunnen telers sneller en doelgerichter ingrijpen, wat aanzienlijke schade kan voorkomen. Dit leidt op zijn beurt tot een gezondere teelt en een hogere opbrengst.

Een andere belangrijke toepassing is de oogstinschatting. Door beeldanalyses kan een robot inschatten hoeveel opbrengst er te verwachten is, hoe groot de gewassen zijn en in welke mate ze rijp zijn. Dit stelt telers in staat om het oogstmoment nauwkeuriger te plannen en de rijpste gewassen eerst te selecteren.

\subsection{Objectdetectie en segmentatie}
De technieken besproken in de vorige sectie vallen onder de overkoepelende termen objectdetectie en segmentatie. Volgens \textcite{Sharma2020} richt objectdetectie zich op het identificeren en classificeren van objecten in een afbeelding. De precieze locatie van elk object wordt aangeduid met een begrenzingskader. Het beoordelen van rijpheid bijvoorbeeld, zoals hierboven beschreven en weergegeven in figuur \ref{fig:strawberry}, gebeurt met objectdetectie.

\begin{figure}
    \centering
    \includegraphics[width=0.9\textwidth]{strawberry.png}
    \caption[Voorbeeld objectdetectie.]{\label{fig:strawberry}Een visueel voorbeeld van objectdetectie: groen duidt rijpe aardbeien aan, rood onrijpe en blauw halfrijpe \autocite{Chai2023}.}
\end{figure}

Segmentatie daarentegen gaat een stap verder: elk pixel in de afbeelding wordt toegewezen aan een specifieke klasse, waardoor objecten niet alleen worden geïdentificeerd, maar ook volledig worden ingekleurd op pixelniveau in plaats van met een kader. Als voorbeeld brengt het de exacte verspreiding van een plantenziekte in kaart of onderscheidt het een plant van de achtergrond.
 
Segmentatie biedt een gedetailleerdere visuele interpretatie dan objectdetectie. Er zijn twee vormen van segmentatie: semantische segmentatie en instantiesegmentatie. Bij semantische segmentatie worden alle objecten van eenzelfde categorie in dezelfde kleur gemarkeerd. Instantiesegmentatie daarentegen wijst elk individueel object binnen een categorie een unieke kleur toe. Hierdoor kunnen objecten binnen dezelfde klasse van elkaar worden onderscheiden. Figuur \ref{fig:plants} toont beide vormen van segmentatie. De proefopstelling van deze bachelorproef bestaat uit een semantisch segmentatiemodel. 

\begin{figure}
    \centering
    \includegraphics[width=0.65\textwidth]{plants.png}
    \caption[Voorbeeld segmentatie.]{\label{fig:plants}Een visueel voorbeeld van segmentatie. De middelste kolom toont semantische segmentatie van afzonderlijke planten. De rechterkolom geeft instantiesegmentatie weer van individuele bladeren binnen elke plant \autocite{Lei2024}.}
\end{figure}

Hoewel objectdetectie en segmentatie krachtige en veelgebruikte technieken zijn, brengen ze nog steeds uitdagingen met zich mee. Deze komen aan bod in sectie 4, want eerst wordt semantische segmentatie diepgaander besproken.

\section{Semantische segmentatie dieper bekeken}

\section{Uitdagingen bij segmentatie in precisielandbouw}

Uit onderzoek van \textcite{Luo2024} blijkt dat de interesse in deep learning, en meer specifiek in semantische segmentatie, binnen de precisielandbouw groeit. Deze techniek speelt een belangrijke rol in toepassingen gebaseerd op computervisie. Semantische segmentatie maakt het mogelijk om objecten op pixelniveau binnen een afbeelding te classificeren en toe te wijzen aan specifieke klassen. Modellen zoals U-Net, SegNet en DeepLab worden tegenwoordig vaak toegepast in diverse segmentatietoepassingen. Toch blijven segmentatiemodellen binnen complexe landbouwomgevingen op knelpunten stuiten.

De robuustheid van bestaande segmentatiemethoden voor het verwerken van complexe afbeeldingen moet namelijk verbeterd worden. Daarnaast blijft hun vermogen om te generaliseren naar ongeziene data beperkt. Een bijkomend obstakel is het beperkte aantal gelabelde data, wat de training en evaluatie van de deep learning-modellen bemoeilijkt. Segmentatiemethoden die de dataset uitbreiden met augmentatie, zoals het gebruik van synthetische data, stellen deep learning-methoden in staat hun segmentatiemogelijkheden te verbeteren.

Onderzoek van \textcite{Charisis2024} bevestigt dat instantiesegmentatie dezelfde uitdagingen ervaart binnen de precisielandbouw. Zij voegen toe dat er universele prestatiemaatstaven ontbreken, waardoor modelvergelijkingen vaak afhankelijk zijn van onderzoeksspecifieke criteria. Verder zorgt de beperkte beschikbaarheid van open-source semi-automatische labeltools voor tijdsintensieve manuele annotatie van de datasets. Ten slotte vormt de praktische toepassing van het theoretische model in het veld vaak een extra belemmering.

Ondanks deze huidige knelpunten neemt de ontwikkeling van deep learning gebaseerde segmentatie binnen de precisielandbouw snel toe. Er wordt verwacht dat deze technologie de komende jaren een aanzienlijke impact zal hebben.

Bovenvermelde uitdagingen worden verder geanalyseerd binnen de methodologie, waar een breder antwoord wordt gegeven op de bijbehorende deelvraag.

\section{Synthetische data vergaren}
% zeg dat de bachelorproef in hoofdzaak zich focust op het probleem van te weinig (gelabelde) data
\section{Applicaties van synthetische data}
\section{Modeloptimalisatie}

% -------------------------------
\section{Tips}

Dit hoofdstuk bevat je literatuurstudie. De inhoud gaat verder op de inleiding, maar zal het onderwerp van de bachelorproef *diepgaand* uitspitten. De bedoeling is dat de lezer na lezing van dit hoofdstuk helemaal op de hoogte is van de huidige stand van zaken (state-of-the-art) in het onderzoeksdomein. Iemand die niet vertrouwd is met het onderwerp, weet nu voldoende om de rest van het verhaal te kunnen volgen, zonder dat die er nog andere informatie moet over opzoeken \autocite{Pollefliet2011}.

Je verwijst bij elke bewering die je doet, vakterm die je introduceert, enz.\ naar je bronnen. In \LaTeX{} kan dat met het commando \texttt{$\backslash${textcite\{\}}} of \texttt{$\backslash${autocite\{\}}}. Als argument van het commando geef je de ``sleutel'' van een ``record'' in een bibliografische databank in het Bib\LaTeX{}-formaat (een tekstbestand). Als je expliciet naar de auteur verwijst in de zin (narratieve referentie), gebruik je \texttt{$\backslash${}textcite\{\}}. Soms is de auteursnaam niet expliciet een onderdeel van de zin, dan gebruik je \texttt{$\backslash${}autocite\{\}} (referentie tussen haakjes). Dit gebruik je bv.~bij een citaat, of om in het bijschrift van een overgenomen afbeelding, broncode, tabel, enz. te verwijzen naar de bron. In de volgende paragraaf een voorbeeld van elk.

\textcite{Knuth1998} schreef een van de standaardwerken over sorteer- en zoekalgoritmen. Experten zijn het erover eens dat cloud computing een interessante opportuniteit vormen, zowel voor gebruikers als voor dienstverleners op vlak van informatietechnologie~\autocite{Creeger2009}.

Let er ook op: het \texttt{cite}-commando voor de punt, dus binnen de zin. Je verwijst meteen naar een bron in de eerste zin die erop gebaseerd is, dus niet pas op het einde van een paragraaf.

\begin{figure}
    \centering
    \includegraphics[width=0.8\textwidth]{grail.jpg}
    \caption[Voorbeeld figuur.]{\label{fig:grail}Voorbeeld van invoegen van een figuur. Zorg altijd voor een uitgebreid bijschrift dat de figuur volledig beschrijft zonder in de tekst te moeten gaan zoeken. Vergeet ook je bronvermelding niet!}
\end{figure}
%%=============================================================================
%% Methodologie
%%=============================================================================

\chapter{\IfLanguageName{dutch}{Methodologie}{Methodology}}%
\label{ch:methodologie}

%% TODO: In dit hoofstuk geef je een korte toelichting over hoe je te werk bent
%% gegaan. Verdeel je onderzoek in grote fasen, en licht in elke fase toe wat
%% de doelstelling was, welke deliverables daar uit gekomen zijn, en welke
%% onderzoeksmethoden je daarbij toegepast hebt. Verantwoord waarom je
%% op deze manier te werk gegaan bent.
%% 
%% Voorbeelden van zulke fasen zijn: literatuurstudie, opstellen van een
%% requirements-analyse, opstellen long-list (bij vergelijkende studie),
%% selectie van geschikte tools (bij vergelijkende studie, "short-list"),
%% opzetten testopstelling/PoC, uitvoeren testen en verzamelen
%% van resultaten, analyse van resultaten, ...
%%
%% !!!!! LET OP !!!!!
%%
%% Het is uitdrukkelijk NIET de bedoeling dat je het grootste deel van de corpus
%% van je bachelorproef in dit hoofstuk verwerkt! Dit hoofdstuk is eerder een
%% kort overzicht van je plan van aanpak.
%%
%% Maak voor elke fase (behalve het literatuuronderzoek) een NIEUW HOOFDSTUK aan
%% en geef het een gepaste titel.

\begin{listing}
    \begin{minted}{python}
        import pandas as pd
        import seaborn as sns
        
        penguins = sns.load_dataset('penguins')
        sns.relplot(data=penguins, x="flipper_length_mm", y="bill_length_mm", hue="species")
    \end{minted}
    \caption[Voorbeeld codefragment]{Voorbeeld van het invoegen van een codefragment.}
\end{listing}

\begin{table}
    \centering
    \begin{tabular}{lcr}
        \toprule
        \textbf{Kolom 1} & \textbf{Kolom 2} & \textbf{Kolom 3} \\
        $\alpha$         & $\beta$          & $\gamma$         \\
        \midrule
        A                & 10.230           & a                \\
        B                & 45.678           & b                \\
        C                & 99.987           & c                \\
        \bottomrule
    \end{tabular}
    \caption[Voorbeeld tabel]{\label{tab:example}Voorbeeld van een tabel.}
\end{table}

\lipsum[21-25]


% Voeg hier je eigen hoofdstukken toe die de ``corpus'' van je bachelorproef
% vormen. De structuur en titels hangen af van je eigen onderzoek. Je kan bv.
% elke fase in je onderzoek in een apart hoofdstuk bespreken.

%\input{...}
%\input{...}
%...

\input{conclusie}

%---------- Bijlagen -----------------------------------------------------------

\appendix

\chapter{Onderzoeksvoorstel}

Het onderwerp van deze bachelorproef is gebaseerd op een onderzoeksvoorstel dat vooraf werd beoordeeld door de promotor. Dat voorstel is opgenomen in deze bijlage.

%% TODO: 
\section*{Samenvatting}

% Kopieer en plak hier de samenvatting (abstract) van je onderzoeksvoorstel.
AI biedt het potentieel om autonome landbouwtechnologieën verder te moderniseren. De landbouwsector blijkt echter moeilijk toegankelijk voor de implementatie van bestaande AI-oplossingen. Dit komt vooral door de complexiteit en variabiliteit van natuurlijke omgevingen. Een specifieke uitdaging binnen dit domein is de realtime segmentatie van druivenplanten, mede door het tekort aan (gelabelde) datasets. Dit leidt tot de hoofdonderzoeksvraag: \emph{'Hoe kan een deep learning model worden toegepast voor realtime segmentatie van druivenranken en -trossen in wijngaarden, en hoe draagt gesynthetiseerde data bij aan de generalisatie?'} Deze bachelorproef richt zich op de ontwikkeling en training van een segmentatiemodel met data afkomstig uit zowel echte als gesimuleerde wijngaarden. Het model wordt iteratief bijgesteld op basis van evaluaties om de prestaties te optimaliseren. Het einddoel is een model dat geschikt is voor realtime toepassingen. Deze aanpak zal naar verwachting leiden tot een praktisch toepasbaar segmentatiemodel.  Het gebruik van gesynthetiseerde data zal naar veronderstelling de generalisatie van het model positief beïnvloeden. Bovendien bewijst gesynthetiseerde data zijn waarde als aanvullende bron wanneer echte landbouwdata moeilijk beschikbaar is.

% Verwijzing naar het bestand met de inhoud van het onderzoeksvoorstel
%---------- Inleiding ---------------------------------------------------------

% TODO: Is dit voorstel gebaseerd op een paper van Research Methods die je
% vorig jaar hebt ingediend? Heb je daarbij eventueel samengewerkt met een
% andere student?
% Zo ja, haal dan de tekst hieronder uit commentaar en pas aan.

%\paragraph{Opmerking}

% Dit voorstel is gebaseerd op het onderzoeksvoorstel dat werd geschreven in het
% kader van het vak Research Methods dat ik (vorig/dit) academiejaar heb
% uitgewerkt (met medesturent VOORNAAM NAAM als mede-auteur).
% 

\section{Inleiding}%
\label{sec:inleiding}

Twintig jaar geleden was er nauwelijks sprake van Belgische wijnbouw. De laatste jaren is deze sector echter uitgegroeid tot een productieve pijler binnen onze landbouw. Volgens \textcite{FODEconomie2024} werd er in 2023 meer dan 3,4 miljoen liter wijn geproduceerd, en die groei zet zich voort. Ten opzichte van het voorgaande jaar steeg de productie met bijna dertien procent, terwijl het aantal hectaren met elf procent toenam. Daarom krijgt deze sector aandacht binnen het \textit{Flanders AI Research Programme} (FAIR). FAIR is een consortium van onderzoeksgroepen aan Vlaamse universiteiten en onderzoekscentra, gericht op AI-onderzoek in diverse Vlaamse sectoren. Het Instituut voor Landbouw-, Visserij- en Voedingsonderzoek (ILVO) is een van deze centra. Samen met UAntwerpen werken zij aan een usecase om taken autonoom uit te voeren in wijngaarden met behulp van AI, waaronder het monitoren van de druivenkwaliteit. 

Voor dit doel wordt een reinforcement learning (RL)-algoritme ontwikkeld dat op een kleine landbouwrobot wordt geïmplementeerd. De robot heeft een arm en sensoren om druiven te meten. Om een taak autonoom uit te voeren zijn drie controlesystemen nodig: navigatiecontrole, actuatorcontrole en taakcontrole. Het RL-model moet deze controllers samenvoegen tot één geïntegreerde controller zodat de robot zelfstandig beslissingen kan nemen, afhankelijk van de situatie.

Opdat het RL-model een goede beslissing kan nemen, moet duidelijke omgevingsinformatie worden aangereikt. Er is besloten dit te doen in de vorm van een lokale 2D-kaart van de omgeving, opgesteld met Lidar- en camerasensordata. Een mogelijke methode hiervoor is het inkleuren van gekalibreerde puntenwolken met informatie uit gesegmenteerde beelden, om deze vervolgens te projecteren op een vlak parallel aan het grondvlak. Deze bachelorproef draagt bij aan de interpretatie van camerabeelden door semantische segmentatie te realiseren.

Het semantische segmentatiemodel moet in realtime de druivenranken en -trossen kunnen segmenteren. Dergelijke modellen vereisen veel rekenkracht en geheugen. Bovendien presteren ze vaak minder goed in landbouw, omdat hun architectuur niet afgestemd is op de complexiteit van natuurlijke omgevingen en vanwege het beperkte aantal (gelabelde) datasets. Dit leidt tot de hoofdonderzoeksvraag: \emph{'Hoe kan een deep learning model worden toegepast voor realtime segmentatie van druivenranken en -trossen in wijngaarden, en hoe draagt gesynthetiseerde data bij aan de generalisatie?'}

De volgende deelvragen worden in deze bachelorproef behandeld:
\begin{itemize}
    \setlength{\itemsep}{0pt}
    \setlength{\parskip}{0pt}
    \item Wat zijn de kenmerken van een wijngaard?
    \item Welke uitdagingen ondervinden segmentatiemodellen bij toepassing in de landbouw?
    \item Welk model is geschikt voor druivenranken en -trossen?
    \item Op welke manier dient de synthetische data te worden samengesteld om het segmentatiemodel beter te trainen?
    \item Hoe kan modeloptimalisatie plaatsvinden om segmentatie in realtime mogelijk te maken?
\end{itemize}

Het eindresultaat van de bachelorproef is een \textbf{proof-of-concept} van het afgestemde segmentatiemodel voor de robot, dat gewassen in wijngaarden kan segmenteren. De prestaties van het model en de evaluaties binnen het onderzoek tonen aan of synthetische data de generalisatie heeft verbeterd.

%---------- Stand van zaken ---------------------------------------------------

\section{Literatuurstudie}%
\label{sec:literatuurstudie}

\subsection{Uitdagingen van segmentatie binnen precisielandbouw}
Uit onderzoek van \textcite{Luo2024} blijkt dat de interesse in deep learning, en meer specifiek in semantische segmentatie, binnen de precisielandbouw groeit. Deze techniek speelt een belangrijke rol in toepassingen gebaseerd op computervisie. Ze maakt het mogelijk om objecten op pixelniveau binnen een afbeelding te classificeren en toe te wijzen aan specifieke klassen. Modellen, zoals U-Net, SegNet en DeepLab, worden tegenwoordig vaak toegepast in diverse segmentatietoepassingen. Toch blijkt segmentatie binnen complexe landbouwomgevingen nog steeds een uitdaging te zijn. De robuustheid van bestaande segmentatiemethoden voor het verwerken van complexe afbeeldingen moet namelijk verbeterd worden. Daarnaast blijft hun vermogen om te generaliseren naar ongeziene data beperkt. Een bijkomend obstakel is het beperkte aantal gelabelde data, wat de training en evaluatie van nieuw ontwikkelde deep learning-modellen bemoeilijkt. Segmentatiemethoden die de dataset uitbreiden met augmentatie, zoals het gebruik van synthetische data, stellen deep learning-methoden in staat hun segmentatiemogelijkheden te verbeteren.

Onderzoek van \textcite{Charisis2024} bevestigt dat instantiesegmentatie dezelfde uitdagingen ervaart binnen de precisielandbouw. Zij voegen toe dat er universele prestatiemaatstaven ontbreken, waardoor modelvergelijkingen vaak afhankelijk zijn van onderzoeksspecifieke criteria. Verder zorgt de beperkte beschikbaarheid van open-source semi-automatische labeltools voor tijdsintensieve manuele annotatie van de datasets. Ten slotte vormt de praktische toepassing van het theoretische model in het veld vaak een extra uitdaging. Ondanks deze huidige knelpunten neemt de ontwikkeling van deep learning-gebaseerde segmentatie binnen de precisielandbouw snel toe. Er wordt verwacht dat deze technologie de komende jaren een aanzienlijke impact zal hebben.

\subsection{Dataset augmentatie met synthetische data}
Zoals eerder vermeld kan synthetische data bijdragen aan de uitbreiding en verbetering van bestaande datasets om semantische segmentatiemodellen te trainen. \textcite{Anderson2022} onderzochten het effect van synthetische data op de kwaliteit van datasets. Deze data werd gegenereerd op basis van CAD-productiemodellen, met als doel echte auto's te segmenteren. Uit de resultaten bleek dat modellen die uitsluitend met synthetische data getraind werden, een lage mean Intersection over Union (mIoU) hadden op de echte validatiedata. Door echter een klein aantal echte afbeeldingen toe te voegen aan de synthetische dataset, verbeterde de accuraatheid aanzienlijk. Bovendien presteerden de gebruikte modellen (U-net en Double-U-net) beter wanneer hun dataset werd uitgebreid met synthetische data, in vergelijking met modellen die enkel getraind werden op echte data.

\textcite{Anderson2022} stellen ook dat synthetische data het mogelijk maakt om sneller grote hoeveelheden trainingsdata te genereren. Daarbovenop wordt de annotatie van deze data automatisch uitgevoerd, waardoor tijd en kosten worden bespaard. Verder kunnen verschillende aspecten van de afbeeldingen eenvoudig worden aangepast, zoals de omgeving, lichtinval, enzovoort. Deze flexibiliteit is bijzonder waardevol in landbouwtoepassingen, waar onderzoekers verschillende weersomstandigheden en groeifases actief in de datasets moeten opnemen.

\subsection{Modeloptimalisatie}
\textcite{LopezGonzalez2024} stellen dat de toenemende rekencapaciteit van computers heeft geleid tot een trend waarbij convolutionele neurale netwerken (CNN’s) steeds groter en complexer worden. Dit veroorzaakt een sterke toename in het aantal parameters en berekeningen. Zulke netwerken zijn daardoor lastig te gebruiken op apparaten met beperkte middelen, zoals boordcomputers van autonome voertuigen. Om deze modellen te verkleinen zonder nauwkeurigheid te verliezen, is het essentieel om goed geteste modellen te combineren met compressietechnieken.

\textcite{Denil2013} merkten op dat het trainen of toepassen van \textit{transfer learning} op bekende CNN-modellen vaak resulteert in een overbodig aantal feature maps, onnodige berekeningen en hoog geheugengebruik, wat de prestaties nadelig beïnvloedt.

In hun paper introduceren \textcite{LopezGonzalez2024} een reeks optimalisaties voor U-Net, SegNet en DeepLabv3+. Ze combineren geavanceerde \textit{filter pruning} met verfijnde finetuningstrategieën om irrelevante convolutionele filters te identificeren en te verwijderen met behulp van Principal Component Analysis (PCA). In combinatie met een innovatieve scoreverdelingsmethode leidt dit tot aanzienlijk compactere netwerken. Voor U-Net ontwikkelden ze een unieke \textit{layer pruning}-techniek die de impact van encoder-decoder connecties analyseert en overbodige lagen elimineert. Ze rapporteren indrukwekkende resultaten, waaronder een reductie van maximaal 98,7\% in parameters en 97,5\% in floating point operations, zonder merkbare afname in nauwkeurigheid. Hierdoor worden zowel redundantie als rekenlast beperkt, terwijl de prestaties behouden blijven.

%---------- Methodologie ------------------------------------------------------
\section{Methodologie}%
\label{sec:methodologie}

De bachelorproef doorloopt een iteratief proces en richt zich op de implementatie en verfijning van het segmentatiemodel. Het proces begint met het in kaart brengen van de variatie binnen wijngaarden, waarbij relevante kenmerken worden geïdentificeerd. Vervolgens worden camerabeelden verzameld uit echte wijngaarden en data gecapteerd uit het testveld van ILVO. Daarnaast worden gesimuleerde wijngaarden opgezet om aanvullende data te genereren.

Na de dataverzameling worden evaluatiemaatstaven bepaald om de prestaties van het model te beoordelen. De modelontwikkeling begint met het selecteren van potentiële architecturen op basis van wetenschappelijke literatuur, waarna de verzamelde data wordt voorbereid. Het gekozen model wordt eerst getraind met enkel de echte data. Daarna volgt een tweede trainingsronde, waarin eerst synthetische data en vervolgens echte data worden gebruikt om het model verder te verfijnen. Beide versies van het model worden geëvalueerd.

Binnen de iteratieve cyclus ligt de nadruk op het doorvoeren van bijsturingen op basis van de evaluatie, met als doel de prestaties te verbeteren en de sim-to-real gap te verkleinen. Dit kan onder andere inhouden dat de modelarchitectuur wordt aangepast of dat de trainingsdata wordt aangevuld. Na elke bijsturing volgt een evaluatie om de impact van de aanpassingen te meten. Tot slot wordt het model geoptimaliseerd voor realtime toepassingen, waarna een finale evaluatie plaatsvindt om de nauwkeurigheid en efficiëntie onder realtime omstandigheden te bevestigen.

In figuur \ref{fig:met} wordt het verloop visueel weergegeven.

\begin{figure}[h]
    \includegraphics[width=8cm]{img/verloop.png}
    \caption{Visualisatie methodologie.}
    \label{fig:met}
\end{figure}

Voor de technische uitwerking kunnen tools zoals Python, PyTorch, GitHub en Blender worden ingezet; deze lijst is mogelijks niet exhaustief of definitief.

%---------- Verwachte resultaten ----------------------------------------------
\section{Verwacht resultaat, conclusie}%
\label{sec:verwachte_resultaten}

\subsection{Segmentatiemodel}
Het onderzoek beoogt een inzetbaar segmentatiemodel te ontwikkelen dat toepasbaar is op een veldrobot voor het segmenteren van druivenplanten in camerabeelden. Het model wordt beoordeeld op nauwkeurigheid en efficiëntie. De nauwkeurigheid wordt vastgesteld door te kijken hoe goed de segmentaties van het model overeenkomen met handmatige annotaties. Belangrijke evaluatiemaatstaven hiervoor zijn Intersection over Union (IoU), de Dice-coëfficiënt, precisie, recall, de boundary F1-score, pixelnauwkeurigheid en mean Average Precision (mAP). Daarnaast wordt de efficiëntie van het model beoordeeld op inferentietijd, modelgrootte, geheugengebruik en de benodigde rekenkracht voor realtime toepassing.

\subsection{Synthetische traindata}
Er wordt verwacht dat gesynthetiseerde data een merkbaar positief effect zal hebben op de generalisatie van het model. Dit komt doordat het tekort aan echte data wordt aangevuld en een bredere variatie aan omstandigheden wordt geboden. De prestaties van het model, getraind uitsluitend met beperkte echte data, worden vergeleken met die van het model dat zowel met echte als gesynthetiseerde data is getraind. Deze manier toont aan of gesynthetiseerde data een realistische aanvulling vormt op data uit wijngaarden en kan bijdragen aan de generalisatie van het model.

\subsection{Meerwaarde}
Door segmentatie worden gewassen in de camerabeelden ingekleurd. Met behulp van gekalibreerde LIDAR-sensoren kunnen deze ingekleurde druivenranken en -trossen worden weergegeven in de gegenereerde puntwolken. Een 2D-projectie van deze 3D-datapunten vormt een kaart waarop de gewassen duidelijk zijn aangeduid. Dit biedt een visuele weergave van de gewaslocatie om het RL-model te ondersteunen bij het kiezen van acties.

%%---------- Andere bijlagen --------------------------------------------------
% TODO: Voeg hier eventuele andere bijlagen toe. Bv. als je deze BP voor de
% tweede keer indient, een overzicht van de verbeteringen t.o.v. het origineel.
%\input{...}

%%---------- Backmatter, referentielijst ---------------------------------------

\backmatter{}

\setlength\bibitemsep{2pt} %% Add Some space between the bibliograpy entries
\printbibliography[heading=bibintoc]

\end{document}

%%=============================================================================
%% Inleiding
%%=============================================================================

\chapter{\IfLanguageName{dutch}{Inleiding}{Introduction}}%
\label{ch:inleiding}

De inleiding moet de lezer net genoeg informatie verschaffen om het onderwerp te begrijpen en in te zien waarom de onderzoeksvraag de moeite waard is om te onderzoeken. In de inleiding ga je literatuurverwijzingen beperken, zodat de tekst vlot leesbaar blijft. Je kan de inleiding verder onderverdelen in secties als dit de tekst verduidelijkt. Zaken die aan bod kunnen komen in de inleiding~\autocite{Pollefliet2011}:

\begin{itemize}
  \item context, achtergrond
  \item afbakenen van het onderwerp
  \item verantwoording van het onderwerp, methodologie
  \item probleemstelling
  \item onderzoeksdoelstelling
  \item onderzoeksvraag
  \item \ldots
\end{itemize}

Twintig jaar geleden was er nauwelijks sprake van Belgische wijnbouw. De laatste jaren is deze sector echter uitgegroeid tot een productieve pijler binnen onze landbouw. Volgens \textcite{FODEconomie2024} werd er in 2023 meer dan 3,4 miljoen liter wijn geproduceerd, en die groei zet zich verder voort. 

(FAIR, ILVO introduceren / kader van stageplaats) reinforcement learning model heel kort benoemen

foto van treebot - in map graphics? zelf aan te maken

synth data/afbeeldingen zijn in opmars, wordt gebruikt in medische wereld

\section{\IfLanguageName{dutch}{Probleemstelling}{Problem Statement}}%
\label{sec:probleemstelling}

Treebot, de autonome landbouwrobot van ILVO, heeft visuele waarneming nodig om zelfstandig tussen de wijnranken te navigeren. Om een gepaste actie te kiezen, moet de robot kunnen zien waar de druivenranken en -trossen zich bevinden. Computervisie kan dit mogelijk maken, specifiek in de vorm van semantische segmentatie. Hierdoor kan hij de gewassen herkennen en onderscheiden van de achtergrond.

Het semantische segmentatiemodel moet in realtime de druivenranken en -trossen kunnen segmenteren. De implementatie van dergelijke modellen op edge devices, zoals kleine landbouwrobots, vormt een uitdaging vanwege beperkte rekenkracht en geheugen. Daarnaast presteren deze modellen vaak minder goed in een landbouwcontext, omdat hun architectuur niet is afgestemd op de complexiteit en variabiliteit van natuurlijke omgevingen. Het grootste struikelblok blijft echter het beperkte aantal beschikbare en gelabelde landbouwdatasets. Het annoteren van vegetatiebeelden is namelijk een zeer tijdrovend en arbeidsintensief proces. Dit bemoeilijkt de training en generalisatie van een model aanzienlijk.

\section{\IfLanguageName{dutch}{Onderzoeksvraag}{Research question}}%
\label{sec:onderzoeksvraag}

Bovenstaande knelpunten leiden tot de hoofdonderzoeksvraag van deze bachelorproef: \emph{'Hoe kan een deep learning-model worden toegepast voor realtime segmentatie van druivenranken en -trossen in wijngaarden, en hoe draagt synthetische data bij aan de generalisatie?'} 

Om deze onderzoeksvraag te beantwoorden, worden verschillende deelaspecten onderzocht. Deze staan geformuleerd in de volgende deelvragen:

\begin{itemize}
    \setlength{\itemsep}{0pt}
    \setlength{\parskip}{0pt}
    % \item Wat zijn de natuurlijke en visuele kenmerken van een wijngaard?
    \item Welke uitdagingen ondervinden segmentatiemodellen bij toepassing in de landbouw?
    \item Welke modellen zijn geschikt voor druivenranken en -trossen?
    \item Op welke manier dient synthetische data samengesteld te worden om het segmentatiemodel beter te trainen?
    \item Hoe kan modeloptimalisatie plaatsvinden om segmentatie in realtime op edge devices mogelijk te maken?
\end{itemize}

\section{\IfLanguageName{dutch}{Onderzoeksdoelstelling}{Research objective}}%
\label{sec:onderzoeksdoelstelling}

Het eindresultaat is een proof-of-concept van het segmentatiemodel voor de robot, gericht op het herkennen van gewassen in wijngaarden. Daarbij is rekening gehouden met hardwarebeperkingen van de landbouwrobot. Visuele invoer waarop het model werd getraind, bestaat grotendeels uit synthetische beelden. De modelprestaties en evaluatieresultaten in dit onderzoek bepalen in welke mate synthetische data bijdraagt aan een verbeterde generalisatie van het model. Zo wordt duidelijk of synthetische data toegankelijk is voor deep learning-modellen die ingezet worden in wijngaarden bij schaarste aan data en labels.

\section{\IfLanguageName{dutch}{Opzet van deze bachelorproef}{Structure of this bachelor thesis}}%
\label{sec:opzet-bachelorproef}

% Het is gebruikelijk aan het einde van de inleiding een overzicht te
% geven van de opbouw van de rest van de tekst. Deze sectie bevat al een aanzet
% die je kan aanvullen/aanpassen in functie van je eigen tekst.

De rest van deze bachelorproef is als volgt opgebouwd:

In Hoofdstuk~\ref{ch:stand-van-zaken} wordt een overzicht gegeven van de stand van zaken binnen het onderzoeksdomein, op basis van een literatuurstudie.

In Hoofdstuk~\ref{ch:methodologie} wordt de methodologie toegelicht en worden de gebruikte onderzoekstechnieken besproken om een antwoord te kunnen formuleren op de onderzoeksvragen.

% TODO: Vul hier aan voor je eigen hoofstukken, één of twee zinnen per hoofdstuk

In Hoofdstuk~\ref{ch:conclusie}, tenslotte, wordt de conclusie gegeven en een antwoord geformuleerd op de onderzoeksvragen. Daarbij wordt ook een aanzet gegeven voor toekomstig onderzoek binnen dit domein.
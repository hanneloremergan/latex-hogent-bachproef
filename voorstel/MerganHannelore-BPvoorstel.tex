%==============================================================================
% Sjabloon onderzoeksvoorstel bachproef
%==============================================================================
% Gebaseerd op document class `hogent-article'
% zie <https://github.com/HoGentTIN/latex-hogent-article>

% Voor een voorstel in het Engels: voeg de documentclass-optie [english] toe.
% Let op: kan enkel na toestemming van de bachelorproefcoördinator!
\documentclass{hogent-article}

% Invoegen bibliografiebestand
\addbibresource{voorstel.bib}

% Informatie over de opleiding, het vak en soort opdracht
\studyprogramme{Professionele bachelor toegepaste informatica}
\course{Bachelorproef}
\assignmenttype{Onderzoeksvoorstel}
% Voor een voorstel in het Engels, haal de volgende 3 regels uit commentaar
% \studyprogramme{Bachelor of applied information technology}
% \course{Bachelor thesis}
% \assignmenttype{Research proposal}

\academicyear{2024-2025} % TODO: pas het academiejaar aan

% TODO: Werktitel
\title{Semantisch segmentatiemodel voor gewasdetectie in wijngaarden met behulp van gesynthetiseerde data}

% TODO: Studentnaam en emailadres invullen
\author{Hannelore Mergan}
\email{hannelore.mergan@student.hogent.be}

% TODO: Medestudent
% Gaat het om een bachelorproef in samenwerking met een student in een andere
% opleiding? Geef dan de naam en emailadres hier
% \author{Yasmine Alaoui (naam opleiding)}
% \email{yasmine.alaoui@student.hogent.be}

% TODO: Geef de co-promotor op
\supervisor[Co-promotor]{A. Willekens (ILVO, \href{mailto:axel.willekens@ilvo.vlaanderen.be}{axel.willekens@ilvo.vlaanderen.be})}

% Binnen welke specialisatierichting uit 3TI situeert dit onderzoek zich?
% Kies uit deze lijst:
%
% - Mobile \& Enterprise development
% - AI \& Data Engineering
% - Functional \& Business Analysis
% - System \& Network Administrator
% - Mainframe Expert
% - Als het onderzoek niet past binnen een van deze domeinen specifieer je deze
%   zelf
%
\specialisation{Mobile \& Enterprise development}
\keywords{Scheme, World Wide Web, $\lambda$-calculus}

\begin{document}

\begin{abstract}
  Hier schrijf je de samenvatting van je voorstel, als een doorlopende tekst van één paragraaf. Let op: dit is geen inleiding, maar een samenvattende tekst van heel je voorstel met inleiding (voorstelling, kaderen thema), probleemstelling en centrale onderzoeksvraag, onderzoeksdoelstelling (wat zie je als het concrete resultaat van je bachelorproef?), voorgestelde methodologie, verwachte resultaten en meerwaarde van dit onderzoek (wat heeft de doelgroep aan het resultaat?).
\end{abstract}

\tableofcontents

% De hoofdtekst van het voorstel zit in een apart bestand, zodat het makkelijk
% kan opgenomen worden in de bijlagen van de bachelorproef zelf.
%---------- Inleiding ---------------------------------------------------------

% TODO: Is dit voorstel gebaseerd op een paper van Research Methods die je
% vorig jaar hebt ingediend? Heb je daarbij eventueel samengewerkt met een
% andere student?
% Zo ja, haal dan de tekst hieronder uit commentaar en pas aan.

%\paragraph{Opmerking}

% Dit voorstel is gebaseerd op het onderzoeksvoorstel dat werd geschreven in het
% kader van het vak Research Methods dat ik (vorig/dit) academiejaar heb
% uitgewerkt (met medesturent VOORNAAM NAAM als mede-auteur).
% 

\section{Inleiding}%
\label{sec:inleiding}

Twintig jaar geleden was er nauwelijks sprake van Belgische wijnbouw. De laatste jaren is deze sector echter uitgegroeid tot een productieve pijler binnen onze landbouw. Volgens \textcite{FODEconomie2024} werd er in 2023 meer dan 3,4 miljoen liter wijn geproduceerd, en de productie blijft groeien. Het aantal hectaren neemt ook elk jaar toe. Daarom krijgt deze sector aandacht binnen het\textit{ Flanders AI Research Programme} (FAIR). FAIR is een consortium van onderzoeksgroepen aan Vlaamse universiteiten en onderzoekscentra, gericht op AI-onderzoek in diverse Vlaamse sectoren. Het Instituut voor Landbouw-, Visserij- en Voedingsonderzoek (ILVO) is een van deze centra. Samen met UAntwerpen werken zij aan een project om de druivenkwaliteit in wijngaarden autonoom te monitoren met behulp van AI.

Voor dit project wordt een reinforcement learning (RL)-algoritme ontwikkeld, dat op een kleine landbouwrobot wordt geïmplementeerd. De robot is uitgerust met een arm en sensoren om druiven te meten. Hij kan reeds autonoom monitoren op basis van drie controllers: navigatie, besturing en taakuitvoering. Het RL-model moet deze controllers combineren tot één geïntegreerde controller, waardoor de robot zelfstandig beslissingen kan maken, afhankelijk van de situatie.

Een belangrijk knelpunt in dit project is het ontbreken van een omgevingskaart. Het RL-model heeft deze kaart nodig om zijn toestand te kunnen waarnemen en zo de beste acties te bepalen. Een gesegmenteerde 2D-kaart kan hiervoor een oplossing bieden. Een mogelijke aanpak is om gekalibreerde puntenwolken in te kleuren met informatie uit gesegmenteerde beelden en deze punten vervolgens in 2D te projecteren. Deze bachelorproef draagt bij aan de ontwikkeling van deze kaart door de semantische segmentatie van beelden uit wijngaarden te realiseren.

Het semantische segmentatiemodel moet op de robot kunnen draaien en gewassen nauwkeurig segmenteren, wat niet vanzelfsprekend is. Dergelijke modellen vereisen veel rekenkracht en geheugen. Bovendien presteren ze vaak minder goed in landbouwomgevingen vanwege de modelarchitecturen, unieke kenmerken van landbouwafbeeldingen en het gebrek aan (gelabelde) landbouwdatasets. Dit leidt tot de hoofdonderzoeksvraag: \emph{'Hoe kan een deep learning model worden toegepast voor nauwkeurige en efficiënte segmentatie van gewassen in een wijngaard, en hoe draagt gesimuleerde data bij aan de generalisatie?'}

Binnen het probleemgebied wordt onderzocht welke unieke kenmerken wijngaardafbeeldingen hebben en welke uitdagingen deze vormen voor de segmentatie. Daarnaast wordt gekeken naar welk model geschikt is voor segmentatie binnen de context van wijngaarden.

In het oplossingsgebied ligt de focus op praktische modelverbetering. Er wordt onderzocht hoe virtueel gegenereerde data de beperkte echte data kan aanvullen en hoe (semi-)automatische datalabeling kan worden toegepast op beide soorten data. Daarnaast wordt gekeken naar de modeloptimalisatie voor toepassing op de robot.

Het eindresultaat van de bachelorproef is een proof-of-concept van het afgestemde segmentatiemodel voor de robot, dat gewassen in de wijngaard kan segmenteren. Het model zal worden beoordeeld op efficiëntie en nauwkeurigheid.

%---------- Stand van zaken ---------------------------------------------------

\section{Literatuurstudie}%
\label{sec:literatuurstudie}

Hier beschrijf je de \emph{state-of-the-art} rondom je gekozen onderzoeksdomein, d.w.z.\ een inleidende, doorlopende tekst over het onderzoeksdomein van je bachelorproef. Je steunt daarbij heel sterk op de professionele \emph{vakliteratuur}, en niet zozeer op populariserende teksten voor een breed publiek. Wat is de huidige stand van zaken in dit domein, en wat zijn nog eventuele open vragen (die misschien de aanleiding waren tot je onderzoeksvraag!)?

Je mag de titel van deze sectie ook aanpassen (literatuurstudie, stand van zaken, enz.). Zijn er al gelijkaardige onderzoeken gevoerd? Wat concluderen ze? Wat is het verschil met jouw onderzoek?

Verwijs bij elke introductie van een term of bewering over het domein naar de vakliteratuur, bijvoorbeeld~\autocite{Hykes2013}! Denk zeker goed na welke werken je refereert en waarom.

Draag zorg voor correcte literatuurverwijzingen! Een bronvermelding hoort thuis \emph{binnen} de zin waar je je op die bron baseert, dus niet er buiten! Maak meteen een verwijzing als je gebruik maakt van een bron. Doe dit dus \emph{niet} aan het einde van een lange paragraaf. Baseer nooit teveel aansluitende tekst op eenzelfde bron.

Als je informatie over bronnen verzamelt in JabRef, zorg er dan voor dat alle nodige info aanwezig is om de bron terug te vinden (zoals uitvoerig besproken in de lessen Research Methods).

% Voor literatuurverwijzingen zijn er twee belangrijke commando's:
% \autocite{KEY} => (Auteur, jaartal) Gebruik dit als de naam van de auteur
%   geen onderdeel is van de zin.
% \textcite{KEY} => Auteur (jaartal)  Gebruik dit als de auteursnaam wel een
%   functie heeft in de zin (bv. ``Uit onderzoek door Doll & Hill (1954) bleek
%   ...'')

Je mag deze sectie nog verder onderverdelen in subsecties als dit de structuur van de tekst kan verduidelijken.

%---------- Methodologie ------------------------------------------------------
\section{Methodologie}%
\label{sec:methodologie}

Tijdens de \textbf{eerste fase} ligt de focus op het verzamelen, analyseren, simuleren en labelen van data. Echte data wordt verzameld via camera of uit bestaande datasets, terwijl virtuele wijngaarden worden ontworpen in Blender. De visuele kenmerken van wijngaardbeelden komen naar voren. Het labelen van de gewassen zal (gedeeltelijk) automatisch verlopen via Roboflow. Dit leidt tot een gecombineerde dataset van reële en gesynthetiseerde data, specifiek gericht op de segmentatie van druivenplanten. Deze fase is acht weken lang.

De \textbf{tweede fase} richt zich op de ontwikkeling van het model en de impact van gesimuleerde data. Na een literatuuronderzoek naar segmentatie-uitdagingen in wijngaard- en landbouwdata volgt de selectie van een geschikt semantisch segmentatiemodel. Dit model wordt geïmplementeerd met behulp van Keras of GitHub-repositories. Training gebeurt eerst uitsluitend met echte data en wordt vervolgens aangevuld met gesimuleerde data. Naast de algemene modelbeoordeling vindt een evaluatie plaats om te bepalen of virtuele data de generalisatie van het model versterkt. Deze fase resulteert in een afgestemd segmentatiemodel voor gewassen in wijngaarden en neemt vier weken in beslag.

De \textbf{derde en laatste fase} concentreert zich op de optimalisatie van het model voor veldtoepassingen. Met technieken zoals pruning en quantization wordt het model geschikt gemaakt voor gebruik op een robot met beperkte rekenkracht. Een eindbeoordeling van het geoptimaliseerde model sluit het proces af. Deze slotfase duurt twee weken.

%---------- Verwachte resultaten ----------------------------------------------
\section{Verwacht resultaat, conclusie}%
\label{sec:verwachte_resultaten}

Hier beschrijf je welke resultaten je verwacht. Als je metingen en simulaties uitvoert, kan je hier al mock-ups maken van de grafieken samen met de verwachte conclusies. Benoem zeker al je assen en de onderdelen van de grafiek die je gaat gebruiken. Dit zorgt ervoor dat je concreet weet welk soort data je moet verzamelen en hoe je die moet meten.

Wat heeft de doelgroep van je onderzoek aan het resultaat? Op welke manier zorgt jouw bachelorproef voor een meerwaarde?

Hier beschrijf je wat je verwacht uit je onderzoek, met de motivatie waarom. Het is \textbf{niet} erg indien uit je onderzoek andere resultaten en conclusies vloeien dan dat je hier beschrijft: het is dan juist interessant om te onderzoeken waarom jouw hypothesen niet overeenkomen met de resultaten.



\printbibliography[heading=bibintoc]

\end{document}